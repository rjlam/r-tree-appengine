We built a small navigation application on top of the R-tree interface.
Given two points, if they are connected by roads, the application returns the list of rectangles which comprise (approximately) the shortest path.
We say approximately because the ``distance'' between two roads is simply the size of the one you're currently on---this design decision was purely due to time and technical limitations.
What were the reasons for developing this particular application?

An immediate motivation is that navigation applications are a well-known, simple spatial query that is rather interesting.
Moreover, shortest-path algorithms (in our case, Dijkstra's) are extremely well-known, but they are not often presented in a distributed setting.
Implementing Dijkstra's ``on top of'' a Google app-engine R-tree was an educative foray into distributed graph queries.
We already know the graph is extremely well-behaved: the data being spatial means that the distance between nodes obeys many nice constraints, such as the triangle inequality (our distance approximation notwithstanding).
Our algorithm can be understood by two subroutines.
The main one, algorithm \ref{alg:findpath}, takes two points, generates an appropriate graph, and then searches the graph for the best path.
The details involved in ``generating an appropriate graph'' are shown in algorithm \ref{alg:buildgraph}.

\begin{algorithm}
\caption{BuildGraph: An algorithm to build a search graph using an R-tree.}\label{alg:buildgraph}
\KwData{An R-Tree $R$ and two leaves $\ell_s$, $\ell_t$.}
\KwResult{A graph $G$ to help compute a path between $\ell_s$, $\ell_t$.}
$M\longleftarrow\text{ComputeMBR}(s,t)$ // $M$ is the minimum bounding rectangle on the leaves\;
$\mathcal S\longleftarrow\text{Search}(R,M)$ // $\mathcal S$ is the set of rectangles touched by $M$\;
$V\longleftarrow\{\ell\mid\ell\in\mathcal S\}$ // Our graph has vertex-set $V$ indexed by each $\ell\in\mathcal S$\;
$E\longleftarrow\{(u,v)\mid \ell_u\cap\ell_v\neq\emptyset\}$ // If two leaves overlap, they are connected in $G$. This can be efficiently computed through $\text{Search}(R,u)$ calls\;
$W$ maps $(u,v)$ to $\text{area}(u)$: that edge has weight equal to the area of the outgoing road\;
\Return $G=(V,E,W)$\;
\end{algorithm}

\begin{algorithm}
\caption{FindPath: Our basic navigation algorithm}\label{alg:findpath}
\KwData{An R-Tree $R$ and two points $s$ and $t$.}
\KwResult{A set of connected roads of least path-weight.}
Let $\ell_s$ be the leaf in $R$ closest to $s$\;
Let $\ell_t$ be the leaf in $R$ closest to $t$\;
$G\longleftarrow\text{BuildGraph}(R,\ell_s,\ell_t)$\;
\Return $\text{Dijkstra}(G,s,t)$\;
\end{algorithm}

As implied by the algorithm, the graph Dijkstra's algorithm uses is implicit in the data, and only made explicit in the small areas our algorithm actually needs.
It can only make the graph explicit by using the R-tree, and it is from that interaction things become interesting.
We have two core observations:

1) Using the R-tree can help trim the portion of the graph Dijkstra's algorithm must traverse.
This allows our algorithm to have a better grasp of how big the graph will actually be, and so may improve performance.
In production applications, the graphs may be so large that a shortest-path algorithm will have to be carefully engineered to handle the giant input.
Here we see that the R-tree allows for a natural ``trimming'' of the input graph, perhaps helping make it more manageable.
The downside is that it may lead to a slightly non-optimal path, if in fact the best roads involve going ``out of our way'' a bit, but due to the nice properties of spatial data it will not be too bad.
In the extreme case, the \emph{only} paths may be very winding and leave the trimmed area, so our search would fail!
We are saving our user a very long drive, however.

2) Using R-trees leads to a natural \emph{partition} of the graph.
Imagine a slight extension to this algorithm, where we wish to record the number of times a road is traversed.
Thus a counter is associated with each road, and to update a path requires a \emph{write lock}.
As we build our graph from the R-tree, this immediately and naturally partitions the graph, so our thread would only have a lock on some small part of the graph, easily expressed by a minimum bounding rectangles.
To our knowledge there is \emph{no such} locking mechanism so simple for graphs.
Perhaps this is a reasonable way to approach concurrent graph access, a notoriously tricky problem.

Even in this simple illustration we have seen that R-trees have very interesting interactions with compelling applications.
